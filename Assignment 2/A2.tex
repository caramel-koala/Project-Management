%----------------------------------------------------------------------------------------
%	PACKAGES AND OTHER DOCUMENT CONFIGURATIONS
%----------------------------------------------------------------------------------------

\documentclass[paper=a4, fontsize=11pt]{scrartcl} % A4 paper and 11pt font size

\usepackage[T1]{fontenc} % Use 8-bit encoding that has 256 glyphs
\usepackage[english]{babel} % English language/hyphenation

%----------------------------------------------------------------------------------------
%	TITLE SECTION
%----------------------------------------------------------------------------------------

\newcommand{\horrule}[1]{\rule{\linewidth}{#1}} % Create horizontal rule command with 1 argument of height

\title{	
A Study of Scientific Research Methodology \\ % The assignment title
\horrule{2pt} \\[0.5cm] % Thick bottom horizontal rule
}

\author{Antonio Peters} % Your name

\date{\normalsize\today} % Today's date or a custom date

%----------------------------------------------------------------------------------------
%    ACTUAL DOCUMENT
%----------------------------------------------------------------------------------------

\begin{document}

\maketitle % Print the title

%----------------------------------------------------------------------------------------
\section{Introduction}

Research can be defined as a scientific and systematic search for pertinent information on a specific topic. From this definition we can see that the research process is a a rigorous method with a set procedure to produce results. The process starts off by observing and defining an existing problem, formulating a hypothesis or solution, rigorously testing and retesting the proposed solution, obtaining data on the proposed solution, using the data to come to a conclusion about the problem and finally reviewing the hypothesis against the conclusion to determine the success of the research \cite{kothari2004research}. 

\section{The Purpose of Research}

The main objective of research is to answer questions and increase the knowledge in a given area by making use of the scientific procedure, to find a truth which has yet to be discovered or reinforce or disprove a previous truth \cite{kothari2004research}. 

Research builds upon itself in a logical manner and by adding to the knowledge in the field, allows for more research to be done, whether it be for practical applications, or simply for the sake of knowledge itself.

Of course while, research for the sake of knowledge itself at first holds no practical applications, given time almost all knowledge eventually builds to a point where the knowledge can be used to solve or improve a solution to a real world problem \cite{kothari2004research}.

\pagebreak

\section{Research Methods}

Although all scientific research follows a set structure, it can still be broken down further into two broad, but overlapping areas, namely quantitative and qualitative research. 

Qualitative study is rooted in the social sciences and is, generally, based in determining the meaning of the phenomena being studied where only one instance or case is focused on to achieve a result. This is also known as the verstehen approach \cite{newman1998qualitative}.

Quantitative studies are rooted in a more as statistical studies, where the hypothesis is tested in different environments until a generalization can be deduced from the results to within a certain degree of error \cite{kleinbaum1982epidemiologic}.

It is not unheard of for the two methods to be used in conjunction with one another, often called a mixed method, to extrapolate qualitative meaning from quantitative date \cite{creswell2013research}.

\section{Types of Research}

Research can be grouped into several basic types as well as the methods stated previously, often any one research topic can be rooted in more than one of these types. They are descriptive, analytic, applied, fundamental, conceptual and empirical.

Descriptive research involves describing the state of the topic, the researcher has no control and only states what is occurring within the given system; Analytic research however uses the facts given to derive critical evaluations thereof.

Applied research is used to solve a specific problem, usually derived from a general solution; fundamental research on the other hand seeks to find general solutions, usually from specific solutions to problems.

Conceptual research is mainly used by philosophers to develop or reinterpret new ones by making use of abstract ideas to generalize an existing system. Empirical research relies only on the produced data and experimental outcome to derive a result with little or no regard for the system or theory on which the experiment relies \cite{kothari2004research}.

\section{Conclusion}

It has now been shown that research holds many forms, but the main clear point is that no matter what the form, it always strives to increase the knowledge in the area it is used in. The methodology of scientific research is therefore very flexible as a whole, and the most appropriate method then depends on what exactly is being studied by the researcher.

\pagebreak

\bibliographystyle{plain}
\bibliography{biblio}


%---------------------------------------------------------------------------------------- 

\end{document}